\documentclass[a5paper,10pt]{extreport}
\usepackage[english,russian]{babel}
\usepackage[utf8x]{inputenc}
\usepackage{latexsym,mathrsfs}
\usepackage{stmaryrd, enumitem}
\usepackage{amsthm,amsfonts,amssymb,amsmath}
\usepackage{geometry}
\usepackage{tempora}
\usepackage[pdftex]{graphicx}

\geometry{top=17mm}  \geometry{bottom=20mm}
\geometry{left=17mm} \geometry{right=17mm}
\linespread{1.1}
\parindent=5mm

\def\udcK#1{\noindent УДК~{#1}}
\def\titleK#1{\begin{center}{\textbf {#1}}\end{center}}
\def\authorK#1{\begin{center}{#1}\end{center}}
\def\advisorK#1{Научный руководитель -- {#1}}

\newenvironment{abstractK}{}{~\newline\parindent=5mm\rule{5.5cm}{0.3pt}}
\newenvironment{bibliographyK}{\footnotesize \begin{enumerate}[label={[\arabic*]}]}{\end{enumerate}}

\newtheorem{lemma}{Лемма}
\newtheorem{theorem}{Теорема}
\newtheorem{corollary}{Следствие}
\newtheorem{proposition}{Предложение}
\theoremstyle{definition}
\newtheorem{definition}{Определение}
\theoremstyle{definition}
\newtheorem{question}{Вопрос}
\theoremstyle{definition}
\newtheorem{conjecture}{Гипотеза}

%ВАЖНО: Не менять и не добавлять ничего выше этой строки.
%Изменения вносить только внутри окружения \begin{document}\end{document}
\begin{document}
%УДК
\udcK{004.45}
%Название доклада
\titleK{Эффективная реализация сопрограмм в \\управляемой среде исполнения}
%Информация об авторе
\authorK{Е. В. Пантелеев\\
Новосибирский государственный университет}

%Текст тезисов доклада
%Внутри текста можно использовать стандартные команды TeX а также следующие определенные окружения
%Теорема - \begin{theorem}\end{theorem}
%Лемма - \begin{lemma}\end{lemma}
%Следствие - \begin{corollary}\end{corollary}
%Предложение - \begin{proposition}\end{proposition}
%Определение - \begin{definition}\end{definition}
%Вопрос - \begin{question}\end{question}
%Гипотеза - \begin{conjecture}\end{conjecture}
\begin{abstractK}
Сопрограммы (или корутины) с точки зрения пользователя языка программирования - это потоки, которые управляются средой исполнения языка или самим программистом. Преимущество корутин перед потоками операционной системы заключается в потреблении меньшего объема системных ресурсов – адресного пространства и памяти. Кроме того, переключение контекста корутины дешевле, чем потока, поскольку осуществляется средой исполнения языка. 
\par
В наше время усилился запрос разработчиков многопоточных веб-сервисов на поддержку корутин в языках программирования, поскольку они позволяют упростить и ускорить разработку приложения из-за своих преимуществ, перечисленных выше. Потому сопрограммы были реализованы в популярных языках: Go, Kotlin, C\# и других. К сожалению, сейчас не существует нативного решения в языке Java.
Целю работы является создание работающего прототипа корутин в языке Java. Работа проводится на базе виртуальной машины Excelsior Research Virtual Machine. 
\end{abstractK}

%Научный руководитель
\advisorK{канд. физ.-мат. наук, доц. М. А. Бульонков.}
\end{document}