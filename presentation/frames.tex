
\begin{frame}% первый слайд
	\titlepage
\end{frame}

\begin{frame}	
	\begin{figure}[h!]
		\centering
		\begin{subfigure}[b]{0.4\linewidth}
			\includegraphics[width=\linewidth]{images/server.jpg}
			\caption{Серверы.}
		\end{subfigure}
		\hfill
		\begin{subfigure}[b]{0.4\linewidth}
			\includegraphics[width=\linewidth]{images/accelerator.jpg}
			\caption{Ускорители.}
		\end{subfigure}
	\end{figure}
	\par
	Существует множество задач, в которых необходимо обрабатывать много независимых событий.
\end{frame}

\begin{frame}
	\frametitle{Сопрограммы}
	\begin{itemize}
		\item \textbf{Сопрограмма} (c англ. coroutine) - программный модуль, организованный для обеспечения взаимодействия с другими модулями по принципу кооперативной многозадачности.
		
		\item Сопрограммы способны приостанавливать свое выполнение, сохраняя \textit{контекст} 
		(программный стек и регистры), и передавать управление другой.
	\end{itemize}
\end{frame}

\begin{frame}
	\frametitle{Ключевые отличия от потоков ОС}
	Плюсы сопрограмм
	\begin{itemize}
		\item Переключение контекста сопрограммы требует меньше накладных расходов, чем потока.
		\item Как правило меньший размер стека, а значит, потребление памяти так же меньше.
	\end{itemize}

	Минусы
	\begin{itemize}
		\item Сопрограммы не способны исполняться параллельно.
	\end{itemize}
\end{frame}

\begin{frame}{Поддержка в языках программирования}
	
	\begin{figure}
	\centering
	\hfill
	\begin{subfigure}[b]{0.30\linewidth}
		\includegraphics[width=\linewidth, scale=0.35]{images/cpp.jpg}
		\caption{С++20.}
	\end{subfigure}
	\hfill
	\begin{subfigure}[b]{0.24\linewidth}
		\includegraphics[width=\linewidth, scale=0.24]{images/csharp.jpg} 
		\caption{С\#.}
	\end{subfigure}
	\hfill
	\begin{subfigure}[b]{0.35\linewidth}
		\includegraphics[width=\linewidth, scale=0.35]{images/go.jpg}
		\caption{Go.}
	\end{subfigure}

	\end{figure}
\end{frame}

\begin{frame}
	\includegraphics[scale=0.5]{images/loom.jpg}
	
	\begin{itemize}
	\item	Project Loom – проект на базе OpenJDK, целью которого является разработка сопрограмм для языка Java. 
	\item	На данный момент уже доступна ранняя версия проекта.
	\end{itemize}
\end{frame}

\begin{frame}{Цели и задачи.}
	
	Цель: реализация прототипа сопрограмм в Java.
	\par
	Поставленные задачи:
	\begin{itemize}
		\item Разработать тесты для сравнения производительности потоков и сопрограмм.
		\item Реализовать переключение сопрограмм.
		\item Поддержать трассировку ссылок объектов на стеках сопрограмм(???)
		\item Сравнить производительности сопрограмм и потоков.
	\end{itemize}

	Работа проводится на базе Huawei JDK.
\end{frame}

\begin{frame}{Тесты производительности}
	
	Тесты создавались для измерения 2 параметров.
	\begin{itemize}
		\item Скорость переключения контекста.
		\item Потребление памяти.
	\end{itemize}
	Репозиторий с тестами: https://github.com/minium2/coroutines-benchmark
	
\end{frame}

\begin{frame}{Переключение сопрограмм.}
	\includegraphics[scale=0.5]{images/scheme.jpg}
	\par
	Подходы к реализации:
	\begin{itemize}
		\item OpenJDK: копирование стека сопрограммы при переключении.
		\item HuaweiJDK : изменение указателя стека.
	\end{itemize}
\end{frame}

\begin{frame}{Сборка мусора}
	Что-то про сборку мусора
\end{frame}	
	
\begin{frame}{Результаты}
	\begin{itemize}
	\item Создан набор тестов для сравнения производительности потоков и сопрограммаами.
	\item Реализовано переключение контекста сопрограмм.
	\item Разработана трассировка ссылок объектов на стеках сопрограмм(???).
	\item Получены результаты тестов производительности.
	\end{itemize}
\end{frame}

\begin{frame}{Результаты: скорости переключения}
	Ubuntu, x64, 31 Гб ОЗУ
	\par Каждое значение усреднено по 100 измерениям. HuaweiJDK
	\begin{table}[H]
		%\caption{Число переключений корутин}\label{inc-matrix}
		\begin{tabular}{|c|c|c|c|c|c|}
			\hline \multirow{2}{*}{Cопрограмм, шт.} & \multicolumn{2}{|c|}{Число переключений, 1/сек.}    \\
			\cline{2-3}    & Сопрограммы          & Потоки               \\% & Go                        \\
			\hline 100     & 0                    & 0 (-/+ 0)\\% & 18'187'799 (-/+ 219367)   \\
			\hline 1000    & 0                    & 0 (-/+ 0)\\% & 17'934'078 (-/+ 332778)   \\
			\hline 5000    & 909'148 (-/+ 59'328) & 1'553'872 (-/+ 36'832)\\% & 12'892'417 (-/+ 339410)   \\ % 
			\hline 10000   & 0                    & 0 (-/+ 0)\\% & 8'307'791 (-/+ 79652)     \\ % 
			\hline 50000   & 749'523 (-/+ 6118)   & 361'088 (-/+ 7'853)\\% & 5'292'780 (-/+ 121844)    \\ % 
			\hline 
		\end{tabular}
	\end{table}
	
\end{frame}

\begin{frame}{Результаты: потребление памяти}
	Ubuntu, x64, 31 Гб ОЗУ, HuaweiJDK
	\begin{table}[H]
		%\caption{Число переключений корутин}\label{inc-matrix}
		\begin{tabular}{|c|c|c|c|c|c|}
			\hline \multirow{2}{*}{Число сопрограмм, шт.} & \multicolumn{2}{|c|}{Размер физической памяти.}    \\
			\cline{2-3}    & Сопрограммы  & Потоки  \\
			\hline 100     & 34M          & 0       \\
			\hline 1000    & 38M          & 0       \\
			\hline 5000    & 59M          & 0       \\
			\hline 10000   & 85M          & 0       \\
			\hline 50000   & 107M         & 0       \\ 
			\hline 
		\end{tabular}
	\end{table}
\end{frame}


\begin{frame}{Применение сопрограмм.}
	\begin{itemize}
		\item Реализация бесконечных списков, итераторов, генераторов.
		\item Написание асинхронного и неблокирующего кода(???).
		\item Обработка независимых друг от друга событий.
	\end{itemize}
\end{frame}

\begin{frame}{План дальнейших работ} 
	\begin{itemize}
	\item Переделать функцию переключения контекста.
	\item Реализация возможности миграции сопрограмм с одного потока на другой(???).
	\item Синхронизация: поддержка synchronized блоков(???).
	\item Переключение сопрограммы при вызове ввода вывода.
	\end{itemize}
\end{frame}

\begin{frame}
	\begin{center}
		Спасибо за внимание!
	\end{center}
\end{frame}

