\begingroup 
\renewcommand{\section}[2]{\anonsection{Библиографический список}}
\begin{thebibliography}{00}

\bibitem{coroutine}
    Сопрограммы
    [Электронный ресурс] 
    https://en.wikipedia.org/wiki/Coroutine

\bibitem{modula}
	Модула-2
	[Электронный ресурс] 
	https://en.wikipedia.org/wiki/Modula-2
	
\bibitem{thread}
	Поток
	[Электронный ресурс]
	https://ru.wikipedia.org/wiki/Поток\_выполнения
	
\bibitem{jit-java}
	JIT--компиляция
	https://docs.oracle.com/cd/E13150\_01/jrockit\_jvm/jrockit/geninfo/diagnos/underst\_jit.html
	
\bibitem{simula}
	Симула
	[Электронный ресурс]
	https://en.wikipedia.org/wiki/Simula
	
\bibitem{clu}
	Язык программирования CLU
	[Электронный ресурс]
	https://en.wikipedia.org/wiki/CLU\_(programming\_language)
	
\bibitem{colt}
	Библиотека Colt Parallel
	[Электронный ресурс]
	https://sites.google.com/site/piotrwendykier/software/parallelcolt
	
\bibitem{go-context}
	Переключение контекста в языке Go.
	[Электронный ресурс]
	https://github.com/golang/go/blob/master/src/runtime/asm\_amd64.s
	
\bibitem{loom-main}
	Проект Loom
	[Электронный ресурс]
	https://wiki.openjdk.java.net/display/loom/Main
	
\bibitem{glibc-src}
	Исходный код библиотеки Glibc 2.33
	[Электронный ресурс] 
	https://www.gnu.org/software/libc/ % <glibc_home>\sysdeps\unix\sysv\linux\x86_64\swapcontext.S 

\bibitem{linux-api}
	Майкл Керриск. Linux API. Исчерпывающее руководство. — СПб.: Питер, 2018.

\end{thebibliography}
\endgroup

\clearpage
