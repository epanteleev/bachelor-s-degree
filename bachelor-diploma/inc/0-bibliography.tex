\begingroup 
\renewcommand{\section}[2]{\anonsection{Библиографический список}}
\begin{flushleft}
\begin{thebibliography}{}

	
\bibitem{coroutine}
    Melvin E. Conway. Design of a separable transition-diagram compiler // 
    Communications of the ACM. -- 1963. -- Т. 6, № 7.

\bibitem{modula}
	Вирт Н. Программирование на языке Модула-2. Перевод с англ. В. А. Серебрякова, В. М. Ходукина; Под ред. В. М. Курочкина. — М. : Мир, 1987. — 222 с.
	
\bibitem{thread}
	David R. Butenhof. Programming with POSIX Threads. Addison-Wesley
	
\bibitem{jit-java}
	JIT--компиляция 
	[Электронный ресурс] //
	https://docs.oracle.com/cd/E13150\_01/jrockit\_jvm/jrockit/geninfo/diagnos/underst\_jit.html
	
\bibitem{simula}
	Симула
	[Электронный ресурс] //
	https://en.wikipedia.org/wiki/Simula
	
\bibitem{clu}
	Barbara Liskov. CLU Reference Manual (англ.) / edited by G. Goos, J. Hartmanis. — Berlin — Heidelberg — New York: Springer-Verlag, 1981.
	
\bibitem{colt}
	Библиотека Colt Parallel
	[Электронный ресурс] //
	https://sites.google.com/site/piotrwendykier/software/parallelcolt
	
\bibitem{go-context}
	Спецификация языка Go.
	[Электронный ресурс] //
	https://golang.org/ref/spec
	
\bibitem{loom-main}
	Проект Loom
	[Электронный ресурс] //
	https://wiki.openjdk.java.net/display/loom/Main
	
\bibitem{glibc-src}
	Исходный код библиотеки Glibc 2.33
	[Электронный ресурс] //
	https://www.gnu.org/software/libc/ % <glibc_home>\sysdeps\unix\sysv\linux\x86_64\swapcontext.S 

\bibitem{linux-api}
	Майкл Керриск. Linux API. Исчерпывающее руководство. — СПб.: Питер, 2018.

\bibitem{go}
	Язык  программирования  Go. : Пер. с англ. — М. : ООО “И.Д.  Вильямс”, 2016. — 432 с. 

\bibitem{linux-internals}
	Роберт Лав. Ядро Linux: описание процесса разработки = Linux Kernel Development. — 
	3-е изд. — М.: Вильямс, 2012. — 496 с.
	
\bibitem{fiber}
	Роберт Лав. Сопрограммы и фиберы // Linux. Системное программирование. — 2-е изд.. — 
	СПб.: Питер, 2014. — С. 261. — 448 с. 

\bibitem{go-hep}
	Официальный сайт набора библиотек Go-HEP
	[Электронный ресурс] //
	https://go-hep.org/

\bibitem{cpp-coro-use}
	Bruce Belson. A Survey of Asynchronous Programming Using Coroutines in the Internet of Things and Embedded Systems //
	ACM Transactions on Embedded Computing Systems. -- 2019.
	
\bibitem{c-sharp}
	Язык программирования C\#
	[Электронный ресурс] //
	https://docs.microsoft.com/ru-ru/dotnet/csharp/programming-guide/
	
\bibitem{kotlin-lang}
	Venkat Subramaniam. Programming Kotlin. Book version: P1.0—September 2019
	
\bibitem{cpp20-coro}
	Сопрограммы в С++20
	[Электронный ресурс] //
	https://en.cppreference.com/w/cpp/language/coroutines

\bibitem{linux-kernel}
	Документация ядра linux 
	[Электронный ресурс] //
	https://www.kernel.org/doc/html/latest/
	
\bibitem{multitask}
	Дорот В. Л. Вытесняющая многозадачность // Толковый словарь современной компьютерной лексики. — 3 изд.. — БХВ-Петербург, 2004. — С. 143. — 608 с.
	
\bibitem{real-time-lang}
	Янг С. Алгоритмические языки реального времени. Перевод с англ. Л. В. Ухова; Под редакцией В.В Мартынюка
	-- М. : Мир, 1985.
\end{thebibliography}
\end{flushleft}
\endgroup

\clearpage
