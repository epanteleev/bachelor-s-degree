\section{Цели и задачи}
	Целью данной работы является изучение применимости сопрограмм вместо потоков в программах Java. При ее выполнении ключевой задачей становится создание модуля 
	сопрограмм в языке Java.
	Его разработка проводилась на базе Huawei JDK: альтернативной реализации виртуальной машины Java, которая поддерживает компиляцию перед исполнением. 
	Создание сопрограмм в управляемой среде имеет ряд аспектов: 
	\begin{enumerate}[align=left]
		\item Виртуальная машина должна уметь переключать контекст выполнения. Без этого механизма невозможно представить реализацию сопрограмм.
		\item Требуется проводить сборку мусора объектов, чьи ссылки лежат на стеках и в сохраненных регистрах сопрограмм. 
		\item Сопрограммы должны корректно работать в критических секциях, то есть в случае Java внутри блоков кода, помеченных ключевым словом \textbf{synchronized}.
		\item Среда исполнения должна уметь вытеснять сопрограмму, которая инициировала
		блокирующую операцию, и запускать новую. Примером такой операции может послужить
		сетевой ввод-вывод. Как говорилось раннее, это поможет избежать вытеснения потока.
		\item Необходимо уметь корректно обрабатывать исключение, брошенное из сопрограммы. Причем поведение исключения может быть разным: когда исключение
		развернуло весь стек вызовов, оно может быть переброшено в
		несущий поток либо привести к приостановки сопрограммы и вывода ее стека вызовов.
		\item И наконец, Huawei JDK c модулем сопрограмм должен проходить набор тестов JCK
		\footnote{Java Compatibility Kit (JCK) - набор тестов на совместимость c Java.}.
	\end{enumerate}
	Чтобы минимальный прототип сопрограмм заработал в управляемой среде исполнения, достаточно реализовать первые 
	два пункта. Как было показано в обзоре предметной области, существует несколько способов
	переключения контекста сопрограмм. Для определения оптимального варианта необходимо разработать 
	набор тестов на производительность сопрограмм в управляемых средах Go, OpenJDK/Loom, 
	так как на текущий момент не существует подходящих тестов, которые бы измеряли скорость 
	переключения контекста и размер физической памяти в этих языках.
	Требуется так же выявить точные значения этих параметров и для потоков операционной
	системы, чтобы определить, насколько скорость переключения и потребление памяти лучше
	у сопрограмм. Может быть, полученные значения будут равными в пределах погрешности,
	и все отличия, перечисленные в обзоре области, не будут играть роли в реальных программах?
	\par
	В дальнейшем разработанные тесты можно будет использовать для сравнение производительности сопрограмм в
	Huawei JDK от OpenJDK и Go. Так же стоит применить сопрограммы для решения вычислительных задач.
	Для этого потребуется модифицировать библиотеку Colt Parallel, добавив туда возможность производить
	вычисления с помощью сопрограмм. Поскольку эта библиотека довольно велика, то нет смысла
	изменять ее всю. Достаточно применить сопрограммы в алгоритме многопоточного перемножения
	матриц и дискретного Фурье преобразования, поскольку эти примеры будут довольно показательными в плане
	применимости сопрограмм.
	\par
	В итоге, полный список задач, поставленных для достижения поставленной цели выглядит следущим образом:
	\begin{enumerate}[align=left]
	  	\item Разработать тесты для сравнения производительности потоков и сопрограмм.
	  	\item Создать базовый прототип сопрограмм.
	  	\item Сравнить производительность сопрограмм и потоков.
	  	\item Выявить ключевые плюсы использования сопрограмм.
	\end{enumerate}
\clearpage