\section{Реализация модуля}
	Разработка прототипа сопрограмм проводилась на базе Huawei JDK: альтернативной реализации виртуальной машины Java,
	которая поддерживает компиляцию перед исполнением. 
	Создание сопрограмм в управляемой среде имеет ряд аспектов: 
	\begin{enumerate}[align=left]
		\item Виртуальная машина должна уметь переключать контекст выполнения. Без этого механизма невозможно представить реализацию сопрограмм.
		\item Требуется проводить сборку мусора объектов, чьи ссылки лежат на стеках и в сохраненных регистрах сопрограмм. 
		\item Сопрограммы должны корректно работать в критических секциях, то есть в случае Java внутри блоков кода, помеченных ключевым словом \textbf{synchronized}.
		\item Среда исполнения должна уметь вытеснять сопрограмму, которая инициировала
		блокирующую операцию, и запускать новую. Примером такой операции может послужить
		сетевой ввод-вывод. Как говорилось ранее, это поможет избежать вытеснения потока.
		\item Необходимо уметь корректно обрабатывать исключение, брошенное из сопрограммы. Причем поведение 
		исключения может быть разным: когда исключение
		развернуло весь стек вызовов, оно может быть переброшено в
		несущий поток либо привести к приостановке сопрограммы и распечатке ее стека вызовов.
		\item И наконец, Huawei JDK c модулем сопрограмм должен проходить набор тестов JCK
		\footnote{Java Compatibility Kit (JCK) - набор тестов на совместимость c Java.}.
	\end{enumerate}
	Чтобы минимальный прототип сопрограмм заработал в управляемой среде исполнения, достаточно реализовать первые 
	два пункта. Как было показано в обзоре предметной области, существует несколько способов
	переключения контекста сопрограмм. Для определения оптимального варианта был разработан
	набор тестов на производительность сопрограмм в управляемых средах. О их результатах
	речь пойдет позднее. 
	
\clearpage