\section{Цели и задачи}
	Целью данной работы является реализация прототипа сопрограмм в языке программирования Java. Разработка
	проводилась на базе HuaweiJDK: альтернативной реализации виртуальной машины Java, которая поддерживает компиляцию перед
	исполнением. Создание сопрограмм в управляемой среде имеет ряд требований. 
	\begin{enumerate}
		\item Управляемая среда должна уметь переключать контекст выполнения. Без этого механизма
		невозможно представить реализацию сопрограмм.
		\item Виртуальная машина должна уметь проводить сборку мусора объектов, чьи ссылки лежат на стеках и в сохраненных
		регистрах сопрограмм. 
		\item Сопрограммы должны корректно работать в критических секциях, то есть в случае Java кода внутри блоков,
		помеченных ключевым словом "synchronized".
		\item Среда исполнения должна уметь вытеснять сопрограмму, которая инициировала блокирующую операцию, и
		запускать новую. Примером блокирующей операции может послужить сетевой ввод-вывод или блокировка мьютекса.
		Как говорилось раннее, это поможет избежать блокирование потока.
		\item Необходимо уметь корректно обрабатывать исключение, брошенное из сопрограммы. Причем поведение броска
		исключения может быть разным: когда исключение развернуло весь стек вызовов, оно может быть переброшено в
		несущий поток либо привести к приостановки сопрограммы и печати ее стека вызовов.
		\item И наконец, HuaweiJDK c модулем сопрограмм должен проходить набор тестов JCK.
		\footnote{Java Compatibility Kit (JCK) - набор тестов на совместимость (что он еще там проверяет???)}
	\end{enumerate}
	
	Для того, чтобы минимальный прототип сопрограмм заработал в управляемой среде исполнения, достаточно реализовать первые два пункта. Как было показано в обзоре предметной области, существует несколько способов
	переключения контекста сопрограмм. Для того, чтобы выбрать оптимальный вариант, необходимо разработать набор тестов на производительность сопрограмм в управляемых средах Go, Java/Loom, поскольку на текущий момент не
	существует подходящих
	тестов, которые бы измеряли скорость переключения контекста и размер физической памяти, потребляемой сопрограммаами или потоками.
	Полезно так же сравнить производительность сопрограмм и потоков, чтобы доказать утверждение, данное ранее. В дальнейшем, эти же
	тесты можно будет использовать для сравнение производительности сопрограмм в HuaweiJDK от OpenJDK и Go.
	
	В итоге, полный список задач, поставленных для достижения поставленной цели выглядит следущим образом:
	\begin{enumerate}
		\item Разработать тесты для сравнения производительности потоков и различных реализаций сопрограмм в управляемых средах. 
		\item Реализовать переключение сопрограмм.
		\item Поддержать трассировку ссылок объектов на стеках сопрограмм для сборки мусора. 
		\item Сравнить производительность сопрограмм и потоков.
	\end{enumerate}
	  
\clearpage