\section{Выводы}
	В результате данной работы был проведен анализ реализаций сопрограмм в других языках, 
	в частности Go, Кotlin, C\#. Затем был реализован базовый прототип сопрограмм в
	Huawei JDK, который способен переключать контекст и поддерживает сборку мусора объектов, ссылки которых находятся
	на стеках сопрограмм. Затем было проведено измерение производительности с помощью ранее разработанных тестов.
	Созданный в рамках этой работы прототип сопрограмм обходит в скорости переключения
	OpenJDK в 3--8 раз и потребляет меньше физической памяти.
	\par
	Прототип сопрограмм Huawei JDK еще не до конца реализован, что бы тот был применен в вычислительных задачах.
	Поэтому были применены сопрограммы из Loom/OpenJDK вместо потоков Java в вычислительных задачах на
	примере многопоточного перемножения матриц и вычисления дискретного Фурье--преобразования 
	из библиотеки для научных расчетов Colt Parallel. Для этой цели были модифицированны соответствующие 
	фрагменты этой библиотеки, которые отвечают за параллельную обработку.
	Измерения времен выполнения этих вычислительных задач показали, что применение сопрограмм в 
	такого рода задачах возможно. Сопрограммы при перемножении матриц показали себя лучше,
	чем потоки, а в вычислении преобразования Фурье они имеют одинаковый результат в пределах погрешности.
	Поскольку уже сейчас Huawei JDK обходит в скорости переключения OpenJDK, то ,предположительно, можно ожидать
	большего прироста производительности от применения сопрограмм, чем сейчас.
	\par
	Проверка применимости сопрограмм в реальных системах обработки и сбора данных в рамках
	этой работы оказалась невозможной. Однако, проведенные измерения позволяют предположить,
	что благодаря в разы большей скорости переключения, применение сопрограмм Huawei JDK может дать 
	существенный прирост пропускной способности по сравнению с потоками, а это особенно будет заметно
	с ростом числа каналов связи. 
	\par
	Итак, ключевыми отличиями сопрограмм от потоков являются меньшее потребление памяти, благодаря
	динамически увеличивающемуся стеку, и лучшая скорость переключения контекста из--за реализации 
	их в среде исполнения языка, а не внутри ядра операционной системы.
	\par
	В дальнейшем планируется усовершенствовать прототип, добавив возможности синхронизации
	сопрограмм, исполняемых разными потоками одновременно. Также следует реализовать вытеснение сопрограмм, которые
	инициируют блокирующие операции.  
	Затем необходимо будет внедрить сопрограммы в реальную систему сбора и обработки данных для проверки их 
	применимости. Реализация этого проекта на сопрограммаах позволит в дальнейшем получить точные значения
	пропускной способности, насколько сопрограммы будут полезны в конкретном таком случае.
\clearpage
