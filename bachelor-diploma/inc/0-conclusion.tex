\section{Выводы}
	В результате данной работы был разработан набор тестов для сравнения производительности потоков и сопрограмм в
	управляемых средах языка Go и OpenJDK/Loom. Тесты позволяют измерить скорость переключения и
	потребление физической памяти. Был реализован базовый прототип сопрограмм в
	Huawei JDK, который способен переключать контекст и поддерживает сборку мусора объектов, ссылки которых находятся
	на стеках сопрограмм. Затем было проведено измерение производительности с помощью ранее разработанных тестов.
	Созданный в рамках этой работы прототип сопрограмм обходит в скорости переключения
	OpenJDK в 3--8 раз и потребляет меньше физической памяти.
	\par
	Наконец была изучена применимость сопрограмм вместо потоков Java в вычислительных задачах на
	примере многопоточного перемножения матриц и вычисления дискретного Фурье--преобразования 
	из библиотеки для научных расчетов Colt Parallel. Для этой цели были модифицированны соответствующие 
	фрагменты этой библиотеки, которые отвечают за параллельную обработку. В них потоки заменены 
	на сопрограммы практически один к одному.
	Измерения времен выполнения тестов показали, что применение сопрограмм в такого рода задачах с целью
	повышения производительности не имеет смысла. Потоки при перемножении матриц показали себя лучше,
	чем сопрограммы, а в вычислении преобразования Фурье они имеют одинаковый результат в пределах погрешности.
	
	\par
	Итак, ключевыми отличиями сопрограмм от потоков являются меньшее потребление памяти, благодаря
	динамически увеличивающемуся стеку, и лучшая скорость 
	переключения контекста из--за реализации их в среде исполнения языка, а не внутри ядра операционной системы.
	\par
	В дальнейшем планируется усовершенствовать прототип, добавив возможности синхронизации
	сопрограмм, исполняемых разными потоками одновременно. Также следует реализовать вытеснение сопрограмм, которые
	инициируют блокирующие операции. Это поможет избежать лишнего простоя потока, выполняющих код сопрограмм.
	
\clearpage
