\section{Введение}
	В современном мире язык Java используется для создания крупных многопоточных приложений 
	для решения многих задач, в том числе и научных. Начиная с 90-x годов, Java используется для реализации
	систем управления Большим адронным коллайдером и для организации параллельной обработки результатов 
	экспериментов, например, в библиотеке Colt Parallel\cite{colt}. Язык Java выбран из-за того, что он 
	обладает управляемой средой исполнения, a это значительно упрощает разработку. Управляемая среда представляет
	собой вычислительное окружение, которое позволяет настраивать размер доступной стековой и динамической памяти,
	включающее в себя функции сборки мусора, синхронизации потоков и так далее.
	\par
	Традиционно, параллелизм реализуется внутри операционной системы с помощью механизма потоков, 
	которые абстрагируют взаимодействующие между собой независимо работающие задачи. 
	Поток (англ. thread) -- наименьшая единица обработки, исполнение которой может 
	быть запланировано ядром операционной системы\cite{thread}. 
	Модель программирования, которая позволяет нескольким потокам выполняться в рамках одного экземпляра
	исполняемой программой называется многопоточностью. Потоки в контексте одной программы способны
	совместно использовать ее ресурсы. Многопоточная модель предоставляет
	разработчикам абстракцию параллельного выполнения задач. Преимущество многопоточной программы 
	позволяет ей работать быстрее на компьютерах, имеющих несколько процессоров и/или ядер. Из-за этого, с
	помощью потоков можно добится реального параллельного выполнения задач. Поток создается и планируется
	операционной системой, которая с помощью системных вызовов предлагает разработчику прикладных программ
	манипулировать ими. Параллельное программирование требует осторожности со стороны программиста во
	избежание состояния гонки и другого контринтуитивного поведения. 
	Что касается планирования, то в современных операционных системах общего назначения как правило потоки
	планируются посредством вытесняющей многозадачности. Это предполагает, что операционная система принимает
	решение о переключении между задачами по истечении некоторого выделенного кванта времени\cite{multitask}.
	В такой системе каждой выполняемой задаче дается приоритет, в зависимости от которого принимается решение
	о планировании. В отличие от кооперативной многозадачности, управление операционной системе передаётся
	независимо от состояния работающего потока, благодаря чему, в частности, зависшие (ушедшие в бесконечный цикл)
	задачи не блокируют ядро процессора. За счёт регулярного переключения задач 
	также улучшается отзывчивость системы, своевременность освобождения ресурсов, которые больше
	не используются потоком\cite{real-time-lang}.
	\par
	К сожалению, модель потоков имеет ряд минусов. 
	Операционные системы представляют потоки как универсальное средство многозадачности
	для всех ныне существующих языков программирования. Проблема в том, что потоки 
	-- это достаточно "тяжеловесный" механизм: их создание и переключение несет в себе крупные накладные расходы. 
	Это становится особенно заметно c ростом числа потоков в программе.  
	Избежать накладных расходов на использование потоков можно, применяя вместо них сопрограммы. 
	\par
	Сопрограмма (англ. coroutine) — программный модуль, особым образом организованный для обеспечения взаимодействия с
	другими модулями по принципу кооперативной многозадачности\cite{coroutine}. Выполнение сопрограммы может быть
	приостановлено в точках явного планирования и предано другому такому модулю. При этом будет сохранено полное
	состояние сопрограммы: включая стек, значения регистров и счётчик команд. 
	\par
	Сопрограммы уже были применены в наборе библиотек для научных расчетов Go-HEP\cite{go-hep}.
	Библиотеки написаны на языке Go, в котором нельзя использовать потоки. Если необходима 
	альтернатива потокам, то используются сопрограммы. 
	\par
	Целью данной работы является изучение применимости сопрограмм вместо потоков в параллельных системах. 
	В частности, для изучения применяются многопоточные программы Java. Для достижения цели необходимо решить
	следующие задачи:
	\begin{enumerate}[align=left]
		\item Провести анализ реализаций сопрограмм в других языках. Это позволит понять причину, почему 
		сопрограммы потребляют меньше процессорного времени и памяти.
		\item Создать базовый прототип сопрограмм. Его реализация проводилась на базе Huawei JDK.
		\item Сравнить производительность сопрограмм и потоков.
		\item Выявить ключевые плюсы использования сопрограмм.
	\end{enumerate}
	В обзоре предметной области будут рассмотрены отличия потоков от сопрограмм. 
	Затем, будут показаны способы реализаций сопрограмм в
	различных языках программирования, а после описаны этапы выполнения поставленной цели.
\clearpage