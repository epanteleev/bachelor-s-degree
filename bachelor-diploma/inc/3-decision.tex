\section{Описание решения}
	
	\subsection{Разработка тестов}
	Пред началом реализации сопрограмм в Huawei JDK были разработаны тесты
	производительности для языков Go и ранней версией проекта "Loom". 
	Первый тест измеряет скорость переключение контекста сопрограмм в языках Go и
	Java c "Loom". Тест принимает в аргументах число сопрограмм или потоков, которые при каждом запуске
	исполняют фиксированное число переключений. При этом гарантируется, что программа будет исполнена 
	только одним ядром процессора. Каждый тест Go и Java запускается 100 раз и полученные результаты усредняются.
	При их исполнении измеряется RSS\footnote{Resident set size (RSS) — размер памяти, 
	выделенных процессу операционной системой и в настоящее время находящееся в ОЗУ (RAM)}.
	Результаты измерений для Huawei JDK, OpenJDK и языка Go будут представлены в следующих разделах.
	\clearpage
	
	\subsection{Переключение сопрограмм в Huawei JDK}
	В таблице \ref{switch-table0} показаны результаты измерений скорости переключения сопрограмм Go и 
	OpenJDK/Loom\footnote{Все измерения проводились на операционной системе Ubuntu, kernel 4.15, Intel Core i7-8700,
	4.6 ГГц, 32 Гб ОЗУ}. Видно, что сопрограммы из языка Go выигрывают OpenJDK в скорости переключения.
	
	\begin{table}[H]
		\centering
		\caption{Число переключений сопрограмм}\label{switch-table0}
		\textit{\begin{tabular}{|c|c|c|c|c|c|}
				\hline \multirow{2}{*}{Шт.} & \multicolumn{2}{|c|}{Переключений, тыс./сек.}                     \\
				\cline{2-3}               & OpenJDK/"Loom"                         & Go                  	    \\
				\hline 100                & \numprint{1900} $\pm$ 20\phantom{0}    & \numprint{18187} $\pm$ 219 \\
				\hline \numprint{1000}    & \numprint{1775} $\pm$ 20\phantom{0}    & \numprint{17934} $\pm$ 332 \\
				\hline \numprint{5000}    & \numprint{1703} $\pm$ 30\phantom{0}    & \numprint{12892} $\pm$ 339 \\  
				\hline \numprint{10000}   & \numprint{1924} $\pm$ 235   		   & \numprint{8307} $\pm$ 80   \\  
				\hline \numprint{20000}   & \numprint{1863} $\pm$ 217   		   & \numprint{7045} $\pm$ 72   \\ 
				\hline \numprint{30000}   & \numprint{1772} $\pm$ 182   		   & \numprint{6391} $\pm$ 94   \\ 
				\hline \numprint{40000}   & \numprint{1606} $\pm$ 194              & \numprint{5790} $\pm$ 67   \\ 
				\hline \numprint{50000}   & \numprint{1503} $\pm$ 157   & \phantom{0}\numprint{5292} $\pm$ 122  \\  
				\hline 
		\end{tabular}}
	\end{table}
	
	Поэтому, для реализации в HuaweiJDK был выбран подход языка Go. Из-за простоты использования, в раннем
	прототипе для переключения контекста использовались функции из библиотеки glibc getcontext и swapcontext.
	Результат измерения раннего прототипа приведен в таблице \ref{switch-table-huaweiJDK}.

	\begin{table}[H]
		\centering
		\caption{Число переключений сопрограмм}\label{switch-table-huaweiJDK}
		\textit{\begin{tabular}{|c|c|c|c|c|c|}
				\hline \multirow{2}{*}{Шт.} & \multicolumn{1}{|c|}{Переключений, тыс./сек.}                  \\
				\cline{2-2}               & HuaweiJDK                 \\
				\hline 100                & \numprint{1956} $\pm$ 38  \\
				\hline \numprint{1000}    & \numprint{1829} $\pm$ 12  \\
				\hline \numprint{5000}    & \numprint{1578} $\pm$ 39  \\  
				\hline \numprint{10000}   & \numprint{1316} $\pm$ 20  \\  
				\hline \numprint{20000}   &  1226 $\pm$ 8\phantom{0}  \\ 
				\hline \numprint{30000}   &  1068 $\pm$ 7\phantom{0}  \\ 
				\hline \numprint{40000}   &              928 $\pm$ 7  \\ 
				\hline \numprint{50000}   &              881 $\pm$ 5  \\  
				\hline 
		\end{tabular}}
	\end{table}	
	
	Анализ исходного кода функции swapcontext показал, что переключение можно ускорить. 
	Она используется для переключения потоков в операционных системах на ядре Linux.
	Функция getcontext предоставляет информацию о пользовательском контексте, 
	описывающую состояние потока перед активацией обработчика сигнала, в том числе и 
	предшествующую маску сигналов и сохраненные значения регистров, в частности, программный счетчик и указатель
	стека\cite{linux-api}.
	Функция swapcontext предварительно делает системный вызов для сохранения текущей маски сигналов потока,
	в чем нет необходимости при переключении контекста сопрограмм. Это один из факторов, побудивший
	реализовать аналоги функций getcontext, swapcontext внутри Huawei JDK, которые бы учитывали 
	особенности виртуальной машины. На таблице \ref{huaweiJDK-cmp} представлено сравнение результатов 
	скоростей переключения.  
	\begin{table}[H]
	\centering
	\caption{Сравнение числа переключений}\label{huaweiJDK-cmp}
	\textit{\begin{tabular}{|c|c|c|c|c|c|}
			\hline \multirow{2}{*}{Шт.} & \multicolumn{2}{|c|}{Число переключений, тыс./сек.}   \\
			\cline{2-3}    		   		& getcontext/setcontext    & Функции из HuaweiJDK       \\
			\hline 100     		   		& \numprint{1956} $\pm$ 38 & \numprint{12980} $\pm$ 540 \\
			\hline \numprint{1000} 		& \numprint{1829} $\pm$ 12 & \numprint{11420} $\pm$ 694 \\
			\hline \numprint{5000} 		& \numprint{1578} $\pm$ 39 & \phantom{0}\numprint{5875} $\pm$ 183  \\
			\hline \numprint{10000}		& \numprint{1316} $\pm$ 20 & \phantom{0}\numprint{4459} $\pm$ 162  \\ 
			\hline \numprint{20000}		& 1226 $\pm$ 8\phantom{0}  & \numprint{3604} $\pm$ 93   \\ 
			\hline \numprint{30000}		& 1068 $\pm$ 7\phantom{0}  & \numprint{3031} $\pm$ 94   \\ 
			\hline \numprint{40000}		& 928 $\pm$ 7    		   & \numprint{2653} $\pm$ 87   \\ 
			\hline \numprint{50000}		& 881 $\pm$ 5    		   & \numprint{2315} $\pm$ 60   \\ 
			\hline 
		\end{tabular}
	}
	\end{table}
	Видно, что функции из glibc проигрывают в несколько раз при любом количестве сопрограмм.
	\clearpage
	
	\subsection{Сборка мусора}
	Следущим шагом работы стала сборка мусора объектов, расположенных в зоне видимости функций,
	вызванных в сопрограмме. Виртуальная машина Java хранит список начал и вершин всех потоков,
	созданных в процессе работы.
	Это необходимо для того, чтобы при сборке мусора стало возможным нахождение всего
	корневого множества живых объектов. 
	\par
	В случае сопрограмм необходимо повторить данную логику: требуется хранить все адреса начал и вершин в некотором
	буфере. Но в отличии от потоков, нужно еще сохранять регистры приостановленных сопрограмм для корректной сборки. 
	
	\clearpage
	\subsection{Потребление памяти}
	После реализации базового прототипа, наступил этап измерения потребления физической памяти. 
		
	\begin{table}[H]
		\centering
		\caption{Измерение потребления физической памяти}\label{huaweiJDK-mem}
		\textit{\begin{tabular}{|c|c|c|c|c|c|}
				\hline \multirow{2}{*}{Шт.} & \multicolumn{3}{|c|}{Резидентная память}  \\
				\cline{2-4}    & HuaweiJDK   & OpenJDK/"Loom"    & Go        \\
				\hline 100     & 18 Mб       & 130 Mб    		 & 3,040  Mб \\
				\hline 1000    & 22 Mб       & 161 Mб     		 & 3,105 Mб  \\
				\hline 5000    & 32 Mб       & 187 Mб    		 & 3,156 Mб  \\
				\hline 10000   & 37 Mб       & 193 Mб     		 & 3,308 Mб  \\
				\hline 20000   & 45 Mб       & 196 Mб     		 & 3,320 Mб  \\
				\hline 30000   & 49 Mб       & 197 Mб     		 & 3,350 Mб  \\
				\hline 40000   & 51 Mб       & 200 Mб    		 & 3,390 Mб  \\
				\hline 50000   & 57 Mб       & 202 Mб    		 & 3,407 Mб  \\ 
				\hline 
		\end{tabular}}
	\end{table}
	Как видно из таблицы \ref{huaweiJDK-mem}, Huawei JDK обходит альтернативную реализацию Java OpenJDK в потреблении памяти
	на всех измерениях, но проигрывает языку Go. Огромное потребление ОЗУ программы, исполняемой виртуальной машиной OpenJDK,
	можно связать с наличием JIT-компилятора\footnote{Jist--in--time (JIT) компиляция - компиляция программы во время ее
	исполнения. Эта техника позволяет проводить более смелые оптимизации, поскольку JIT--компилятор владеет большей
	информацией о выполняемой программе, чем статический.} в памяти во время выполнения процесса.
	
	\begin{table}[H]
		\centering
		\caption{Сравнение потребления памяти сопрограмм и потоков.}\label{thread-mem}
		\textit{\begin{tabular}{|c|c|c|c|c|c|}
				\hline \multirow{2}{*}{Шт.} & \multicolumn{2}{|c|}{Размер физической памяти}  \\
				\cline{2-3}    & Сопрограммы   & Потоки    \\
				\hline 100     & 18 Mб         & 34 Mб     \\
				\hline 1000    & 22 Mб         & 35 Mб     \\
				\hline 5000    & 32 Mб         & 37 Mб     \\
				\hline 10000   & 37 Mб         & 40 Mб     \\
				\hline 20000   & 45 Mб         & 49 Mб     \\
				\hline 30000   & 49 Mб         & 56 Mб     \\
				\hline 40000   & 51 Mб         & 63 Mб     \\
				\hline 50000   & 57 Mб         & 72 Mб     \\
				\hline 
		\end{tabular}}
	\end{table}
 	В таблице \ref{thread-mem} представлены результаты измерения потребления памяти потоками операционной системы.
 	Как и говорилось ранее, потоки тратят больший объем ОЗУ, чем сопрограммы. Причины этому были рассмотрены в обзоре
 	предметной области.
 	
 	\clearpage
 	\subsection{Применение сопрограмм в вычислительных задачах}
 	Параллельные системы обработки данных играют большую роль в вычислительных задачах.
 	Потому было бы правильным решением проверить применимость сопрограмм для решения такого
 	рода задач. 
 	\par Для этого был модифицирован фрагмент библиотеки Colt Parallel, отвечающий за 
 	многопоточное перемножение матриц больших размеров (свыше 1000 элементов) и вычисление
 	дискретного преобразования Фурье. В библиотеке можно выделить отдельную часть, которая
 	отвечает за параллельное выполнение задач. Она же и была переписана с использованием
 	сопрограмм, доступных в ранней версии проекта "Loom". В таблице \ref{mat-mul} представлены
 	результаты измерения времени перемножения двух матриц размером 6000x7000 и 7000x6000 соотвественно.
 	Каждый элемент имеет тип double. Замеры проводились на операционной системе CentOS ядро версии 4.18.0 c
 	процессором Intel (R) Xeon(R) Gold 6130, частота которого 2.10 ГГц.  
 
 	%таблица будет перемерена 
 	\begin{table}[H]
 	\begin{center}
 		\caption{Время перемножения матриц.}\label{mat-mul}
 		\begin{tabular}{ |c|c| } 
 			\hline
 			 Потоки  & Сопрограммы   \\
 			\hline
 			 18.3 \pm 0.3 сек.   & 24.0 \pm 5.6 сек.\\ 
 			\hline
 		\end{tabular}
 	\end{center}
	\end{table}
	Как видно, потоки имеют лучший результат, чем сопрограммы. Это связано с тем,
	что при перемножении матриц потоки не приостанавливаются при блокирующих операциях и не взаимодействуют
	между собой. Другими словами, в этой задаче не осуществляется частого переключения потока управления,
	в чем сопрограммы имеют преимущество перед потоками как это было показано ранее. 
 	%эта тоже
  	\begin{table}[H]
 	\begin{center}
 		\caption{Время вычисления дискретного Фурье преобразования.}\label{mat-mul}
 		\begin{tabular}{ |c|c| } 
 			\hline
 			Потоки               & Сопрограммы       \\
 			\hline
 			77.2 \pm 0.3 мсек.   & 77.4 \pm 0.4 мсек.\\ 
 			\hline
 		\end{tabular}
 	\end{center}
	\end{table}
	Однако в задаче вычисления Фурье преобразования время выполнения программы на сопрограммаах и потоках одинаково
	в пределах погрешности.
	\par
	Переделать фрагменты библиотеки Colt Parallel на сопрограммы оказалось гораздо проще, чем ожидалось. С точки 
	зрения программы, сопрограммы и потоки операционной системы это одна и таже сущность. Потому в библиотеке
	потоки были подменены. Сложность заключалась лишь в том, что алгоритмы ограничивали количество нитей числом
	доступных ядер, что не нужно в случае сопрограмм.
	
\clearpage
