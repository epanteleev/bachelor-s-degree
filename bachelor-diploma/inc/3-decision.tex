\anonsection{Описание решения}
	Пред началом реализации сопрограмм в Excelsior RVM были разработаны тесты производительности для языков Go и ранней
	версией проекта "Loom".
	
	\begin{table}[H]
		\caption{Число переключений корутин}\label{inc-matrix}
		\begin{tabular}{|c|c|c|c|c|c|}
			\hline \multirow{2}{*}{Число сопрограмм, шт.} & \multicolumn{3}{|c|}{Число переключений, шт/мин}    \\
			\cline{2-4}    & Go      & Java потоки & Java(Loom) \\
			\hline 100     & 8213552 & 6550504     & 4010587    \\
			\hline 1000    & 8213552 & 2143438     & 3897875    \\
			\hline 5000    & 3958044 & 1440942     & 3868023    \\
			\hline 10000   & 2736352 & 1343995     & 3418569    \\
			\hline 100000  & 2562853 & 888052      & 3889537    \\
			\hline 1000000 & 2135474 & 1089360     & 3227888    \\
			\hline 
		\end{tabular}
	\end{table}
	
	Устройство корутин 
	Реализация сопрограмм в Excelsior RVM отличается от подхода в проекте "Loom". Вместо копирования 
	стека сопрограммы при каждом переключении используется механизм, похожий на переключения потоков 
	операционной системы.
	
		\begin{table}[H]
		\caption{Число запросов к эхо серверу}\label{inc-matrix}
		\begin{tabular}{|c|c|c|c|c|c|}
			\hline \multirow{2}{*}{Соединений, шт.} & 
			\multicolumn{3}{|c|}{Число запросов, мл. шт/мин} \\
			\cline{2-4} & Go   & Java потоки  & Java(Loom)   \\
			\hline 100  & 11,5 & 9,7          & 7,1          \\
			\hline 1000 & 7,8  & 9,5          & 5,7          \\
			\hline 2000 & 7,6  & 9,3          & 5,2          \\
			\hline 3000 & 7,4  & -            & 4,6          \\
			\hline 4000 & 7,2  & -            & 4,3          \\
			\hline 5000 & 7,1  & -            & 4,6          \\
			\hline 
		\end{tabular}
	\end{table}
	
	
\clearpage