\anonsection{Аннотация}

Сопрограммы (или корутины) с точки зрения пользователя языка программирования - это потоки, которые управляются средой исполнения языка или самим программистом. Преимущество корутин перед потоками операционной системы заключается в потреблении меньшего объема системных ресурсов – адресного пространства и памяти. Кроме того, переключение контекста корутины дешевле, чем потока, поскольку осуществляется средой исполнения языка. 
\par
В наше время усилился запрос разработчиков многопоточных веб-сервисов на поддержку корутин в языках программирования, поскольку они позволяют упростить и ускорить разработку приложения из-за своих преимуществ, перечисленных выше. Потому сопрограммы были реализованы в популярных языках: Go, Kotlin, C\# и других. К сожалению, сейчас не существует нативного решения в языке Java.
Целю работы является создание работающего прототипа корутин в языке Java. Работа проводится на базе виртуальной машины Excelsior Research Virtual Machine. 

\clearpage

