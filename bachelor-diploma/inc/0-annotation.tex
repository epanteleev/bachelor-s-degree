\section{Аннотация}

Сопрограммы (англ. coroutine) — программный модуль, особым образом организованный для обеспечения взаимодействия с
другими модулями по принципу кооперативной многозадачности\cite{coroutine}. Выполнение сопрограммы может быть
приостановлено в точках явного планирования и предано другому модулю. 
При этом будет сохранено полное состояние сопрограммы (включая стек, значения регистров и счётчик команд). 
Отличие сопрограмм от потоков операционной системы заключается в потреблении меньшего объема системных ресурсов – адресного пространства и памяти. Кроме того, переключение контекста сопрограммы требует меньших издержек, чем потока,
поскольку осуществляется в пользовательском пространстве операционной системы средой исполнения языка. 
\par
Сопрограммы используются для реализации генераторов, итераторов, бесконечных списков, конечные автоматов внутри одной
подпрограммы. Так же их можно использовать для упрощения написания асинхронного и неблокирующего кода.
Сопрограммы будут полезны разработчикам многопоточных web-сервисов, поскольку позволяют уменьшить время отклика
приложения из-за своих преимуществ, перечисленных выше. 
Они уже были реализованы во многих популярных языках программирования, таких как Go, Kotlin, C\#, но на момент
написания работы сопрограммы не реализованы в языке Java.
Целю является создание работающего прототипа сопрограмм в языке программирования Java. Работа проводилась на базе
HuaweiJDK, альтернативной реализации виртуальной машины Java. Для достижения поставленной цели необходимо было решить следующие задачи:
\begin{itemize}
	\item Разработать тесты для сравнения производительности потоков и сопрограмм.
	\item Реализовать переключение сопрограмм.
	\item Поддержать трассировку ссылок объектов на стеках сопрограмм для сборки мусора. 
	\item Сравнить производительность сопрограмм и потоков.
\end{itemize}

\clearpage

