\section{Аннотация}

Сопрограммы (англ. coroutine) — программный модуль, особым образом организованный для
обеспечения взаимодействия с другими такими модулями по принципу кооперативной
многозадачности\cite{coroutine}. Выполнение сопрограммы может быть
приостановлено в точках явного планирования и предано другому модулю. 
При этом будет сохранено полное состояние сопрограммы (включая стек, значения регистров и счётчик команд). 

\par
Сопрограммы используются для реализации генераторов, итераторов, бесконечных списков. 
Они уже были реализованы во многих популярных языках программирования, таких как Go, Kotlin,
C\#, но не в языке Java. Сопрограммы по своему поведению очень похожи на потоки.
Теоретически, с их помощью можно реализовывать параллелизм на уровне виртуальной машины.
Изучению этого обстоятельства и посвящена работа, целью которой является изучение
применимости сопрограмм вместо потоков в программах Java.
Для достижения поставленной цели необходимо было решить следующие задачи:
\begin{itemize}
	\item Разработать тесты для сравнения производительности потоков и сопрограмм.
	\item Создать базовый прототип сопрограмм.
	\item Сравнить производительность сопрограмм и потоков.
	\item Оценить целесообразность использования сопрограмм.
\end{itemize}
Работа проводилась на базе Huawei JDK, альтернативной реализации виртуальной машины Java, которая способна компилировать код перед исполнением программы (ahead-of-time compilation). 
\par
В ходе работы было выяснено, что отличие сопрограмм от потоков операционной системы
заключается в потреблении меньшего объема системных ресурсов – адресного пространства и
памяти. Что касается переключения контекста, то сопрограммы требуют меньших издержек, чем
потоки. Все из-за того, что переключение осуществляется в пользовательском пространстве
операционной системы средой исполнения языка. 
\par
Сопрограммы будут полезны разработчикам систем параллельной обработки данных из-за своих преимуществ, перечисленных выше.
\clearpage

