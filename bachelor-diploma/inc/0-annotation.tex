\section{Аннотация}

Сопрограммы (или корутины) — программный модуль, особым образом организованный для обеспечения взаимодействия с другими модулями по принципу кооперативной многозадачности \cite{coroutine}. Выполнение корутины может быть приостановлено в определённой точке и предано другой сопрограмме. 
При этом будет сохранено полное состояние сопрограммы (включая стек, значения регистров и счётчик команд). 
Преимущество корутин перед потоками операционной системы заключается в потреблении меньшего объема системных ресурсов – адресного
пространства и памяти. Кроме того, переключение контекста сопрограммы требует меньших издержек, чем потока, поскольку осуществляется в пользовательском пространстве операционной системы. 
\par
Сопрограммы используются для реализации генераторов, итераторов, бесконечных списков, конечные автоматов внутри одной
подпрограммы. Но с недавнего времени разработчики поняли, что сопрограммы можно использовать для написания
асинхронного и неблокирующего кода. Корутины очень полезны разработчикам многопоточных веб-сервисов , поскольку они 
позволяют уменьшить время отклика приложения из-за своих преимуществ, перечисленных выше. Потому сопрограммы 
были реализованы во многих популярных языках программирования, таких как Go, Kotlin, C\#.
К сожалению, сейчас не существует нативного решения для языка Java.
Целю работы является создание работающего прототипа корутин в языке Java. Работа проводилась на базе виртуальной машины Excelsior Research Virtual Machine. Для достижения поставленной цели необходимо было решить следующие задачи:
\begin{itemize}
	\item Разработать тесты для сравнения эффективности различных реализаций корутин.
	\item Реализовать поддержку сопрограмм в Excelsior RVM. 
	\item Сравнить производительность с другими языками, используя ранее разработанные метрики.
\end{itemize}

\clearpage

